\documentclass[
	12pt, % Default font size, values between 10pt-12pt are allowed
	%letterpaper, % Uncomment for US letter paper size
	%spanish, % Uncomment for Spanish
]{fphw}

% Template-specific packages
\usepackage[american]{babel}
\usepackage{inputenc} % Required for inputting international characters
\usepackage[T1]{fontenc} % Output font encoding for international characters
\usepackage{mathpazo} % Use the Palatino font
\usepackage[scheme=plain]{ctex}
\usepackage{graphicx,subfigure} % Required for including images
\usepackage{booktabs} % Required for better horizontal rules in tables
\usepackage{listings} % Required for insertion of code
\usepackage[framed,numbered,autolinebreaks,useliterate]{mcode} %matlab代码
\usepackage{fontspec}
\usepackage{enumerate} % To modify the enumerate environment
\usepackage{amsmath}
\usepackage{amssymb}
\usepackage{mathptmx}
\usepackage{bm}
%----------------------------------------------------------------------------------------
\setmainfont{Times New Roman}

% 代码模板
\lstdefinestyle{mystyle}{
    basicstyle=\ttfamily\footnotesize,
    breakatwhitespace=false,         
    breaklines=true,                 
    keepspaces=true,                 
    numbers=left,                    
    numbersep=5pt,                  
    showspaces=false,                
    showstringspaces=false,
    tabsize=4
}
\lstset{style=mystyle}
% 使用方法:\lstinputlisting[language=Python]{code/a2.py}
%----------------------------------------------------------------------------------------
%	ASSIGNMENT INFORMATION
%----------------------------------------------------------------------------------------
\begin{document}
\title{Homework \#5} % Assignment title
\author{Zhang San, 张三 156413(xuehao)} % Student name and ID
\date{November 12, , 2020} % Due date
\includegraphics[scale=0.75]{img/logo.png}
%\institute{SOUTHERN UNIVERSITY\\OF SCIENCE AND TECHNOLOGY} % Institute or school name
\class{Modern Control and Estimation } % Course or class name
\professor{Prof. San } % Professor or teacher in charge of the assignment

%----------------------------------------------------------------------------------------

\setlength{\abovecaptionskip}{-0.1cm}
\setlength{\belowcaptionskip}{0cm}   %调整图片标题与下文距离


\maketitle % Output the assignment title, created automatically using the information in the custom commands above

%----------------------------------------------------------------------------------------
%	ASSIGNMENT CONTENT
%----------------------------------------------------------------------------------------

\section*{Question 1}
\begin{problem}
	\includegraphics[width=440pt]{img/logo.png}
\end{problem}

%------------------------------------------------

\subsection*{Answer}
\begin{enumerate}[(\itshape a\normalfont)]
    \item From the Question 1(a), we have the $P(S_A)=0.8$ and $P(S_B)=0.6$, the $S_A$ and $S_B$ make decisions independently means that $P(S_AS_B)=P(S_A)P(S_B)$, So we have all the possible outcomes are 
\end{enumerate}
%----------------------------------------------------------------------------------------
\clearpage
\section*{Question 2}
\begin{problem}
	\includegraphics[width=440pt]{img/logo.png}
\end{problem}
%------------------------------------------------
\subsection*{Answer}
\begin{enumerate}[(\itshape a\normalfont)] % Sub-questions styled as italic letters
	\item From the question 2, we have the conditional mean $$E(Y|X=i)=\sum_{j=0}^i j\cdot P(Y=j|X=i)$$, and the
\end{enumerate}

%----------------------------------------------------------------------------------------
\clearpage
\section*{Question 3}
\begin{problem}
    \includegraphics[width=440pt]{img/logo.png}\\
    \includegraphics[width=440pt]{img/logo.png}
\end{problem}
%------------------------------------------------
\subsection*{Answer} 
\begin{enumerate}[(\itshape a\normalfont)]
    \item From sub-question (a), we can get the triangle $S=\{(x,y):-6<y<x<6\}$ showing in the fig 1. 
\end{enumerate}
\end{document}
